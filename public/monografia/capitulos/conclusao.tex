% ----------------------------------------------------------
% Conclusão
% ----------------------------------------------------------
\chapter{Conclusão ou Considerações Finais}

Este trabalho enfatiza que organizações, produtos ou sistemas precisam evoluir para permanecer no mercado. Conforme discutido na fundamentação teórica \ref{sec:hist_industria} sobre a industrialização, sendo que as ferramentas iniciais, embora rústicas, cumprem suas tarefas básicas e a partir delas, é possível realizar aperfeiçoamentos e criar soluções simplificadas que garantem a facilidade na manutenção da operacionalidade do sistema.

De fato, as estruturas já desenvolvidas pelo framework CodeIgniter proporcionam simplificação e otimização dos códigos. Esse modelo de desenvolvimento é possível graças ao ORM, que é uma técnica projetada para facilitar a interação entre o código de programação na pasta model com o bancos de dados instanciado como a \textbf{vtornos}. Isso torna o desenvolvimento mais eficiente, intuitivo e de fácil manutenção.

É importante destacar que o sistema V Tornos contará com ``ajustes mais finos'' a partir da sua implementação inicial. Este sistema tem como base da V Tornos, servindo como uma bússola para guiar a construção de um modelo operacional que poderá ser ampliado com a adição de novos campos ao longo do tempo. A medida que o sistema se desenvolve, ele se torna mais estratégico para a gestão organizacional da empresa. Essa abordagem se fundamenta no conceito de MVP, conforme descrito anteriormente.

O desenvolvimento deste sistema contribui para a presença digital da V Tornos em suas operações predominantes, permitindo uma exploração mais eficaz das tendências do mercado, além de facilitar a classificação dos maquinários. Dessa forma, a empresa poderá adquirir ferramentas adequadas com base nas capacidades dos equipamentos, pois sabendo quais são as maquinas que são mais reparada, pode a gestão investir em equipamentos e peças com mais segurança comparado ao cenário onde não havia o sistema V Tornos.

Vale ressaltar que, com a pandemia de 2020; COVID, a empresa V Tornos desenvolveu peças genéricas para suprir a demanda resultante da escassez de componentes no mercado. Essa adaptação destaca a importância de uma produção local capaz de responder rapidamente a desafios globais. Finalizasse este trabalho com uma reflexão desenvolvida pelo autor no Capítulo 12, ``Desafios e Perspectivas da Indústria Brasileira Rumo à Quarta Revolução Industrial, na seção A Produção Será Local", página 164 \cite{b:industria_v4_2018}. 

\begin{citacao}
Sobre as oportunidades de negócios, se você pensa em um nicho de mercado no qual gostaria de entrar, pergunte a si mesmo: “será que teremos isso no futuro?”, e se a resposta for sim, como você poderá fazer isso acontecer mais cedo? Se não for com o seu celular, esqueça a ideia. 
\end{citacao}