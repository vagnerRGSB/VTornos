%% Preambulo LaTeX: Define classes e características do documento
%% Definição do docuemnto
\documentclass[
	%article,			% Define que este será um artigo (e não uma tese/monografia/relatório)
	12pt,				% Fonte: 12pt
	oneside,			% Impressão: oneside = 1 face, twoside = 2 faces (frente-e-verso)
    %openright,			% capítulos começam em página ímpar (use apenas se usar "twoside")
	a4paper,			% Tamanho do Papel: A4
    chapter=TITLE,		% Todos os capítulos devem ficam em caixa alta
    section=TITLE,		% Todas as seções devem ficar em caixa alta
	english,			% Adiciona Idioma para Hifenização: Inglês
    %spanish,			% Adiciona Idioma para Hifenização: Espanhol
    %french,			% Adiciona Idioma para Hifenização: Francês
	brazil				% Adiciona Idioma para Hifenização: Português do Brasil (o último idioma se torna o principal do documento)
]{abntex2}				% Utilizar ABNTeX2





%% Tipografia
%% Abra este arquivo e selecione uma das opções de fonte nele. A padrão é Times.
\input{configuracoes/tipografia}



%% Pacotes usados pelo documento (se não entender não mexa, hehehe)
\usepackage{courier}                    % Permite a utilização da fonte Courier (para códigos-fonte)
\usepackage[T1]{fontenc}				% Seleção de códigos de fonte.
\usepackage[utf8]{inputenc}				% Codificação do documento (conversão automática dos acentos)
\usepackage{indentfirst}				% Indenta o primeiro parágrafo de cada seção.
\usepackage{nomencl} 					% Usado pela Lista de símbolos
\usepackage{color}						% Controle das cores
\usepackage{graphicx}					% Inclusão de gráficos
\usepackage{float}						% Melhorias para posicionamento de gráficos e tabelas
\usepackage{microtype} 					% Melhorias na justificação
\usepackage{lastpage}   		        % Dá acesso ao número da última página do documento
\usepackage{booktabs}					% Comandos para tabelas
\usepackage{multirow, array}			% Múltiplas linhas e colunas em tabelas
\usepackage{titlesec}                   % Permite criar múltiplas sub seções
\usepackage[table,xcdraw]{xcolor}       % Cores para algumas tabelas especiais
\usepackage[brazilian,hyperpageref]{backref}	 % Inclui nas Referências as páginas onde há as citações
\usepackage{simplecd}                   % Pacote para gerar capa do CD
\usepackage[final]{pdfpages}            % Pacote para incluir um PDF dentro de outro (ficha catalográfica)



%% Adiciona as alterações do ABNTeX-IFFar
\usepackage{abntex-iffar/abntex-iffar}

% % Modificações do ABNTeX para o IFPI
% \usepackage{abntex-iffar/tikz-uml}	    % Pacote Tikz UML para criar UML no LaTeX

%Código php <EM DESENVOLVIMENTO>
\usepackage{xcolor}
\usepackage{listings}

\definecolor{codegreen}{rgb}{0,0.6,0}
\definecolor{codegray}{rgb}{0.5,0.5,0.5}
\definecolor{codepurple}{rgb}{0.58,0,0.82}
\definecolor{codered}{rgb}{1,0,0}
\definecolor{backcolour}{rgb}{0.95,0.95,0.92}
\definecolor{strongblue}{rgb}{0.0,0.0,0.8} % Azul mais forte
\definecolor{codeviolet}{rgb}{0.5,0,0.5} % Roxo

\lstdefinestyle{phplisting}{
    backgroundcolor=\color{backcolour},   
    commentstyle=\color{codegreen},
    keywordstyle=\color{blue}\bfseries,
    numberstyle=\tiny\color{codegray},
    stringstyle=\color{codered},
    basicstyle=\ttfamily\footnotesize,
    breakatwhitespace=false,         
    breaklines=true,                 
    captionpos=b,                    
    keepspaces=true,                 
    numbers=left,                    
    numbersep=5pt,                  
    showspaces=false,                
    showstringspaces=false,
    showtabs=false,                  
    tabsize=2,
    language=PHP,
    morekeywords={class, extends, protected, bool, false, true, public, private, function , static, use, namespace }, % Palavras-chave em azul mais forte
    keywordstyle=[1]\color{strongblue}, % Palavras-chave em azul forte
    morekeywords=[2]{for, if, else, return}, % Adicionando estruturas de controle
    keywordstyle=[2]\color{codeviolet}, % Definindo o estilo roxo
    morekeywords=[3]{array_sum}, % Outras palavras-chave
    keywordstyle=[3]\color{blue}\bfseries, % Palavras-chave adicionais em azul regular
}



%% Metadados
\input{configuracoes/pdf}



%% Metadados
\input{configuracoes/metadados}



%% Configuração do "Citado nas páginas"
\input{configuracoes/citacoes}



%% Cores
\input{configuracoes/cores}



%% Espaçamentos
\input{configuracoes/espacamentos}
