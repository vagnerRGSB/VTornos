%% Preambulo LaTeX: Define classes e características do documento
%% Definição do docuemnto
\documentclass[
	%article,			% Define que este será um artigo (e não uma tese/monografia/relatório)
	12pt,				% Fonte: 12pt
	oneside,			% Impressão: oneside = 1 face, twoside = 2 faces (frente-e-verso)
    %openright,			% capítulos começam em página ímpar (use apenas se usar "twoside")
	a4paper,			% Tamanho do Papel: A4
    chapter=TITLE,		% Todos os capítulos devem ficam em caixa alta
    section=TITLE,		% Todas as seções devem ficar em caixa alta
	english,			% Adiciona Idioma para Hifenização: Inglês
    %spanish,			% Adiciona Idioma para Hifenização: Espanhol
    %french,			% Adiciona Idioma para Hifenização: Francês
	brazil				% Adiciona Idioma para Hifenização: Português do Brasil (o último idioma se torna o principal do documento)
]{abntex2}				% Utilizar ABNTeX2





%% Tipografia
%% Abra este arquivo e selecione uma das opções de fonte nele. A padrão é Times.
%% Tipografia / Fontes
%% AVISO: Todas essas fontes são *bastante semelhantes* aos nomes com as quais as descrevo. Entenda: são iguais, só que oficialmente com outro nome.

%% %%%%%%%%%%%%%%%%%%%%%%%%%%%%%%%%%%%%%%%%%%%%%%%%%%%%% %%
%% Comente todas as outras fontes que você não vai usar! %%
%% %%%%%%%%%%%%%%%%%%%%%%%%%%%%%%%%%%%%%%%%%%%%%%%%%%%%% %%

%% Latin Modern (fonte padrão do LaTeX, Computer Modern, mas com suporte a caracteres acentuados)
%% Considerada a mais clássica e bonita
%\usepackage{lmodern}



%% Times
%% Considerada a mais confortável de ler quando impresso
%%\usepackage{mathptmx}

%% Variação da mesma fonte, com minúsculas diferenças entre uma e outra (coisas bastante técnicas como kerning, aliasing e afins) - Essa tem revisões frequentes
%\usepackage{newtxtext} \usepackage{newtxmath}



%% Arial
%% Considerada mais confortável de ler num computador
%% ** Oficialmente recomendada pelo manual de formatação do IFPI **
\usepackage{helvet} \renewcommand{\familydefault}{\sfdefault}



%% Palatino
%% Uma opção mais elegante à Times
%\usepackage{newpxtext}



%% Kepler
%% Variação evoluída da Palatino, com várias pequenas diferenças e refinamentos
%\usepackage{kpfonts}



%% Libertine
%% Uma fonte estilo Serif comum no Linux
%\usepackage{libertine} %\usepackage[libertine]{newtxmath}



%% Pacotes usados pelo documento (se não entender não mexa, hehehe)
\usepackage{courier}                    % Permite a utilização da fonte Courier (para códigos-fonte)
\usepackage[T1]{fontenc}				% Seleção de códigos de fonte.
\usepackage[utf8]{inputenc}				% Codificação do documento (conversão automática dos acentos)
\usepackage{indentfirst}				% Indenta o primeiro parágrafo de cada seção.
\usepackage{nomencl} 					% Usado pela Lista de símbolos
\usepackage{color}						% Controle das cores
\usepackage{graphicx}					% Inclusão de gráficos
\usepackage{float}						% Melhorias para posicionamento de gráficos e tabelas
\usepackage{microtype} 					% Melhorias na justificação
\usepackage{lastpage}   		        % Dá acesso ao número da última página do documento
\usepackage{booktabs}					% Comandos para tabelas
\usepackage{multirow, array}			% Múltiplas linhas e colunas em tabelas
\usepackage{titlesec}                   % Permite criar múltiplas sub seções
\usepackage[table,xcdraw]{xcolor}       % Cores para algumas tabelas especiais
\usepackage[brazilian,hyperpageref]{backref}	 % Inclui nas Referências as páginas onde há as citações
\usepackage{simplecd}                   % Pacote para gerar capa do CD
\usepackage[final]{pdfpages}            % Pacote para incluir um PDF dentro de outro (ficha catalográfica)



%% Adiciona as alterações do ABNTeX-IFFar
\usepackage{abntex-iffar/abntex-iffar}

% % Modificações do ABNTeX para o IFPI
% \usepackage{abntex-iffar/tikz-uml}	    % Pacote Tikz UML para criar UML no LaTeX

%Código php <EM DESENVOLVIMENTO>
\usepackage{xcolor}
\usepackage{listings}

\definecolor{codegreen}{rgb}{0,0.6,0}
\definecolor{codegray}{rgb}{0.5,0.5,0.5}
\definecolor{codepurple}{rgb}{0.58,0,0.82}
\definecolor{codered}{rgb}{1,0,0}
\definecolor{backcolour}{rgb}{0.95,0.95,0.92}
\definecolor{strongblue}{rgb}{0.0,0.0,0.8} % Azul mais forte
\definecolor{codeviolet}{rgb}{0.5,0,0.5} % Roxo

\lstdefinestyle{phplisting}{
    backgroundcolor=\color{backcolour},   
    commentstyle=\color{codegreen},
    keywordstyle=\color{blue}\bfseries,
    numberstyle=\tiny\color{codegray},
    stringstyle=\color{codered},
    basicstyle=\ttfamily\footnotesize,
    breakatwhitespace=false,         
    breaklines=true,                 
    captionpos=b,                    
    keepspaces=true,                 
    numbers=left,                    
    numbersep=5pt,                  
    showspaces=false,                
    showstringspaces=false,
    showtabs=false,                  
    tabsize=2,
    language=PHP,
    morekeywords={class, extends, protected, bool, false, true, public, private, function , static, use, namespace }, % Palavras-chave em azul mais forte
    keywordstyle=[1]\color{strongblue}, % Palavras-chave em azul forte
    morekeywords=[2]{for, if, else, return}, % Adicionando estruturas de controle
    keywordstyle=[2]\color{codeviolet}, % Definindo o estilo roxo
    morekeywords=[3]{array_sum}, % Outras palavras-chave
    keywordstyle=[3]\color{blue}\bfseries, % Palavras-chave adicionais em azul regular
}



%% Metadados
%% Configurações dos metadados do PDF
\makeatletter
\hypersetup{
  pdftitle={\@title}, 
  pdfauthor={\@author},
  pdfsubject={\@title},
  pdfcreator={LaTeX, abntex2, {abnTeX\-ifpi}},
  %% Coloque aqui suas palavras-chave, cada uma entre chaves: {palavra}{palavra}{outra palavra}...
  pdfkeywords={palavra 1}{palavra 2}{palavra 3}{palavra 4}{palavra 5},
  colorlinks=true,			% Visual dos Links: false = caixas; true = colorido
  linkcolor=cor-link,		% Cor dos Links Internos (preto)
  citecolor=cor-link,		% Cor de Links para Bibliografia (preto)
  filecolor=cor-link,		% Cor para Links a Arquivos (preto)
  urlcolor=cor-link,		% Cor para Links a URLs (preto)
  bookmarksdepth=4
}
\makeatother



%% Metadados
%% %%%%%%%%%%%%%%%%%%%%%%%%%%%%%%%%%%%%%%%%%%%%%%%% %%
%% Metadados do trabalho
%% AVISO: Todos esses dados serão automaticamente convertidos para caixa alta onde necessário
%% %%%%%%%%%%%%%%%%%%%%%%%%%%%%%%%%%%%%%%%%%%%%%%%% %%

%% Título
\titulo{Sistema V. Tornos: Solução Web para Orçamentos e Serviços em Manutenção de Maquinários agrícola e pecuária}

%% Autor
\autor{Vagner Roballo Garcia}

%% Nome do Curso (usado para a Capa do CD)
\nomedocurso{Bacharelado em Sistemas de Informação}

%% Local de publicação
\local{São Borja, Rio Grande do Sul}

%% Preâmbulo do trabalho
\preambulo{Trabalho de Conclusão de Curso (monografia) apresentado como exigência parcial para obtenção do diploma do Curso de Bacharelado em Sistemas de Informação do Instituto Federal de Educação, Ciência e Tecnologia Farroupilha, Campus São Borja.}

%% Orientador
%% "M\textsuperscript{e}." = Abreviação oficial para "Mestre"
\orientador{Prof. Dr. Fernando Luis de Oliveira}
\coorientador{Prof. Dr. Rafael Baldiati Parizi}

%% Tipo de Trabalho
%% - Monografia
%% - Tese (Mestrado)
%% - Tese (Doutorado)
%% - Relatório técnico
\tipotrabalho{Monografia}

%% Data do Trabalho
\data{2024}

%% Nome da Instituição (para a capa)
\instituicao{INSTITUTO FEDERAL DE EDUCAÇÃO, CIÊNCIA E TECNOLOGIA FARROUPILHA
\\
CAMPUS SÃO BORJA
\\
BACHARELADO EM SISTEMAS DE INFORMAÇÃO}

%% Primeiro membro da banca examinadora
\membroum{Prof. Dr. Claiton Correa}

%% Segundo membro da banca examinadora
\membrodois{Prof. Ms. Icaro Lins Iglesias}

%% Terceiro membro da banca examinadora
%\membrotres{Prof. Dr. Xxxxxx Xxxxx}

%% Data da apresentação do trabalho
%% Se não souber a data da apresentação, utilize \underline{\hspace{3.5cm}}
%% Isso cria um sublinhado de 3.5cm, onde você pode escrever a data depois!
%\dataapresentacao{02 de Outubro de 2023}
\dataapresentacao{\underline{\hspace{1.0cm}}/\underline{\hspace{1.0cm}}/\underline{\hspace{1.75cm}}}



%% Configuração do "Citado nas páginas"
%% Configuração das Citações

%% Estilo
%\usepackage[num]{abntex2cite}			% Citações numéricas
\usepackage[alf, 
 versalete, 
 abnt-emphasize = bf, 
abnt-etal-list = 3,
abnt-etal-text = it, 
abnt-and-type = &, 
abnt-last-names = abnt, 
abnt-repeated-author-omit = no  '____.']{abntex2cite}			% Citações "AUTOR, ano"

% Definição do negrito

%% Colocar entre parênteses ou colchetes?
%% Padrão: Parênteses
%% * Fica mais agradável usar colchetes quando se usa citações numéricas
%\citebrackets[]							% Comente essa linha e o documento usará parênteses


%% Configura o "Citado nas Páginas ..." nas referências
%% Não mexa nesse:
\renewcommand{\backref}{}

%% Esse é o texto do "Citado nas páginas ..."
\renewcommand*{\backrefalt}[4]{
	\ifcase #1
		Nenhuma citação no texto.
	\or
		Citado na página #2.
	\else
		Citado #1 vezes nas páginas #2.
	\fi}



%% Cores
%% Cores do Documento

%% Cor dos Links do PDF
%% Usando preta você "esconde" os links
\definecolor{cor-link}{RGB}{0,0,0}

%% Usando azul os links ficam visíveis (ruim para impressão)
%\definecolor{cor-link}{RGB}{8,40,75}



%% Cor para os quadros
%% Dê preferência a cores escuras.
%% Boa referência para cores: https://material.io/guidelines/style/color.html#color-color-palette
\definecolor{cor-quadro}{RGB}{5,28,63}		% Azul Escuro



%% Espaçamentos
%% Espaçamentos
%% O tamanho do parágrafo é dado por:
\setlength{\parindent}{1.5cm}

%% Espaçamento entre um parágrafo e outro:
%% O abntex diz: "tente também \onelineskip"
\setlength{\parskip}{0cm}
