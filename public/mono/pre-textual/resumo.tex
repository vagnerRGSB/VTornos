%% Resumo
\begin{resumo}
Apresentação concisa dos pontos relevantes do documento. Deve Informar ao leitor 
finalidades, metodologia, resultados e conclusões do documento, de tal forma que 
este possa, inclusive, dispensar a consulta ao original. Deve-se usar o verbo na voz 
ativa e na terceira pessoa do singular, contendo de 150 a 500 palavras. O resumo 
deve ser composto de uma sequência de frases concisas, afirmativas e não de 
enumeração de tópicos. Recomenda-se o uso de parágrafo único, mesma fonte do 
trabalho, e espaçamento entrelinhas 1,5. Resumo resumo resumo resumo resumo 
resumo resumo resumo resumo resumo resumo resumo resumo resumo resumo 
resumo resumo resumo resumo resumo resumo resumo resumo resumo resumo 
resumo resumo resumo resumo resumo resumo resumo resumo resumo resumo 
resumo ( ASSOCIAÇÃO BRASILEIRA DE NORMAS TÉCNICAS, 2021).
\vspace{\onelineskip}
\noindent

\textbf{Palavras-chaves}: Palavra 1; Palavra 2; 
Palavra 3.
\end{resumo}